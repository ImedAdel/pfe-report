\chapter*{General Introduction}
\addcontentsline{toc}{chapter}{General Introduction}
\markright{General Introduction}{}

Software is either slow, hard, or ugly---and sometimes all three.
Nonetheless, since the introduction of the modern computer, software has taken the world by storm.
It quickly became an essential component of every business since it paved the way for higher productivity and more automation, and therefore, increased profits.

Nearly forty years \parencite{noauthor_macintosh_2010} have passed since the launch of the Apple Macintosh---one of the first commercially successful \parencite{polsson_chronology_2009} mass-produced personal computers featuring a graphical user interface.
Yet, software is still as inaccessible and inadequate for most users as ever.
Perhaps the best example for such inaccessibility is the fact that this document is being produced using \LaTeX---a fractured software system that requires a plethora of tools to be installed for the sake of producing a legible and aesthetically pleasing document.

Along with the domination of personal computers and software companies in the world, another technology was on the rise---the internet. Since the dotcom bubble in the early 2000s, the internet has reshaped our lives.
Be it entertainment, communication, education, or work, the internet is the primary and most powerful medium.
Therefore, it is no surprise that most software companies switched to \acrfull{saas}, that is hosted software served through the medium of the internet \autocite{december_2019_what_nodate}, which quickly evolved into collaborative software aimed at teams rather than individual users.

The continued sprawl of these technological advancements led to the rise of remote work---a movement that erases any geographical limits and allows businesses and institutions to expand well beyond their headquarters. This movement has been recently magnified to unprecedented levels due to the global pandemic. The main traits of work and education instantly changed and non-collaborative software fell behind to give room to collaborative \acrshort{saas}.

Within these changing dynamics, managing data is still an unsolved problem. Setting, managing, and securing a database is still one of the hardest tasks of building a business. Connecting the database to the rest of the business' applications is not as easy as one might expect. Keeping all employees on-board and managing access to the database, while allowing everyone to seamlessly collaborate is not easily achievable. Requiring all the above while keeping the costs low is impossible. Software geared towards managing data and content is either hard to configure and hard to use or slow to load and slow to on-board.

Based on the belief that software must be accessible, collaborative, fast, and hopefully enjoyable to use, we set out to develop a modern alternative. Using the latest technological innovations, our project is pushing the limits for what is possible with collaborative \acrshort{saas} for managing data and content. Merebase---our project---is a collaborative visual database \acrshort{saas} that challenges the norms, democratizes access to data management software, and fills the need for a no-code and low-cost database software.

Our work is discussed in four chapters.
\begin{itemize}
	\item \nameref{chap:intro} is an introduction to our work and it is intended to add the right context for our project.
	\item \nameref{chap:analysis} starts with exploring the existing solutions and proposing a better alternative.
	\item \nameref{chap:conceptual} is our third chapter. It starts with a deep dive into real-time collaboration and its algorithms, followed by an exploration of the common architectural patterns of modern applications, and eventually, presents our choices. Finally, within this chapter we introduced our detailed architecture of the application, with class and sequence diagrams to the rescue.
	\item \nameref{chap:implementation} is where we went deeper into our technology stack. We objectively compared our options and picked the right tools for the job. Finally, we brought our application into life.
\end{itemize}