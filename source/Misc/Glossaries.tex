\makeglossaries
\newacronym{saas}{SaaS}{Software as a Service}
\newacronym{cms}{CMS}{Content Management Software}
\newacronym{sdk}{SDK}{Software Development Kit}
\newacronym{fdd}{FDD}{Feature-Driven Development}
\newacronym{rbac}{RBAC}{Role-Based Access Control}
\newacronym{ui}{UI}{User Interface}
\newacronym{ux}{UX}{User Experience}
\newacronym{crdt}{CRDT}{Conflict-free Replicated Data Type}
\newacronym{cmrdt}{CmRDT}{Commutative Replicated Data Type}
\newacronym{cvrdt}{CvRDT}{Convergent Replicated Data Type}
\newacronym{ot}{OT}{Operational Transformation}
\newacronym{woot}{WOOT}{WithOut Operational Transformation}
\newacronym{p2p}{P2P}{Peer-To-Peer}
\newacronym{json}{JSON}{JavaScript Object Notation}
\newacronym{mvc}{MVC}{Model-View-Controller}
\newacronym{mvvm}{MVVM}{Model-View-ViewModel}
\newacronym{dom}{DOM}{Document Object Model}
\newacronym{sspl}{SSPL}{Server Side Public License}
\newacronym{rdmbs}{RDMBS}{Relational Database Management System}
\newacronym{waiaria}{WAI-ARIA}{Web Accessibility Initiative – Accessible Rich Internet Applications}
\newacronym{otp}{OTP}{One-Time Password}
\newacronym{gpu}{GPU}{Graphics Processing Unit}

\newglossaryentry{cacheable}
{
	name=cacheable,
	description={A cacheable response is an HTTP response that is stored to be retrieved and used later, saving a new request to the server~\autocite{noauthor_cacheable_nodate}.}
}

\newglossaryentry{polyfill}
{
	name=polyfill,
	description={A piece of code used to provide modern functionality on older browsers that do not natively support it~\autocite{noauthor_polyfill_nodate}.}
}